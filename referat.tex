\documentclass[12pt]{report}
\usepackage{ucs}
\usepackage[utf8]{inputenc}
\usepackage[russian]{babel}
\usepackage{graphicx}
\title{Введение}
\author{Алексей Борисов}
\begin{document}
\chapter*{Введение}
В геодезической практике часто возникает необходимость преобразования координат из одной системы в другую. К примеру, координаты пунктов, вычисленные по результатам спутниковых измерений, относятся к декартовым геоцентрическим координатам в системе ПЗ-90 (при использовании ГЛОНАСС) или WGS-84 (при использовании GPS). Однако нередко требуется предоставление вычисленных координат в других системах --- например, в СК-42, основанной на эллипсоиде Красовского.
\par
Настоящая работа посвящена следующим темам:
\begin{enumerate}
\item Рассмотрены системы координат, используемые в геодезии, и эллипсоиды, на которых они основаны.
\item Рассмотрены способы преобразования координат --- как <<внутреннего>>, то есть из одного типа в другой в пределах одного эллипсоида, так и <<внешнего>>, то есть преобразование координат одного типа из одной системы в другую. 
\item Приведён пример практического использования рассмотренных в п. 2 способов перевода координат. Приложен код программы, принимающей на входе геодезические координаты пункта (B, L, H) в системе WGS-84 и преобразующей эти координаты в различные иные системы. 
\end{enumerate}
\chapter*{1. Системы координат, используемые в геодезии}
В настоящее время в геодезии используются следующие системы координат:
\begin{enumerate}
\item Прямоугольная геоцентрическая,
\item Эллипсоидальная,
\item Плоская прямоугольная.
\end{enumerate}
Рассмотрим эти системы координат подробнее.
\par
Начало координат O \textbf{прямоугольной геоцентрической} системы координат находится в центре масс Земли, ось Z направлена вдоль оси вращения Земли, ось X совмещена с линией пересечения экватора и нулевого меридиана, ось Y дополняет систему координат до правой. Также эта система координат называется \textbf{общеземной}, покольку в ней определяют положение пунктов на всей поверхности Земли.
\par
Если система координат введена для определения положения точек на части земной поверхности, например, на территории одного государства, то её начало O может быть смещено относительно центра масс Земли. В этом случае говорят о \textbf{референцной} системе координат.
\par
Из-за неизбежных ошибок измерений при практическом задании общеземной системы возможно несовпадение её начала с центром масс Земли и повороты осей. В связи с этим существуют несколько реализаций общеземной геоцентрической системы координат и возникает необходимость перехода от одной системы к другой. Кроме того, порой существует необходимость перехода от какой-либо общеземной системы к референцной, принятой в данной местности. Способы перехода (преобразования) координат из одной системы в другую рассмотрены ниже.
\par
Геодезическая эллипсоидальная система координат B, L, H имеет связана с элипсоидом вращения, чьи размеры подбираются для наилучшего соответствия фигуры квазигеоида для Земли в целом либо для какого-то отдельного её участка. \textbf{Геодезическая широта B} точки --- это угол между нормалью, опущенной из данной точки и плоскостью экватора. \textbf{Геодезическая долгота L} --- это двугранный угол между плоскостью меридиана, проходящего через заданную точку и плоскостью нулевого меридиана. \textbf{Геодезическая высота H} --- это отрезок нормали, опущенной из заданной точки между самой точкой и плоскостью эллипсоида.
\par
\begin{figure}[h]
\center{\includegraphics[width=0.8\linewidth]{sks.png}}
\caption{Геоцентрическая и эллипсоидальная системы координат.}
\label{ris:image1}
\end{figure}
\par
Проекцию Гаусса-Крюгера, к которой относится плоская прямоугольная система координат, получают, проецируя земной эллипсоид на поверхность цилиндра, касающегося Земли по какому-либо меридиану. Чтобы искажения длин линий не превышали пределов точности масштаба карты, проецируемую часть земной поверхности ограничивают меридианами с разностью долгот $6^\circ$ (для составления планов в масштабах 1:5~000 и крупнее --- $3^\circ$). Такой участок называется зоной. На рис.~\ref{ris:image2} слева показана <<нарезка>> Земного эллипсоида на зоны, справа --- отдельно взятая зона.
\begin{figure}[h]
\center{\includegraphics[width=0.8\linewidth]{gauss.png}}
\caption{Плоская прямоугольная система координат проекции Гаусса.}
\label{ris:image2}
\end{figure}
\par
Средний меридиан каждой зоны называется осевым. Счёт зон ведётся от Гринвичского меридиана на восток.
\par
После развёртывания цилиндра в плоскость осевой меридиан зоны и экватор изобразятся взаимно-перпендикулярными линиями. Эти изображения принимают за оси зональной системы прямоугольных координат с началом в точке их пересечения. Экватор является осью ординат, осевой меридиан --- осью абсцисс.
\par
Для всех точек на территории России абсциссы положительны. А для того, чтобы ординаты всех точек в пределах одной зоны также были положительны, ординату начала координат каждой зоны принимают равной 500~км.
\chapter*{2. Преобразование координат}
В настоящей главе будут рассмотрены два вида преобразования координат: <<внутреннее>> преобразование из одной системы в другую  (например, из эллипсоидальной в геоцентрическую) и <<внешнее>> (например, преобразование геоцентрических координат на эллипсоиде WGS-84 в геоцентрические координаты на эллипсоиде Красовского в системе СК-42).
\section*{<<Внутреннее>> преобразование координат}
Рассмотрим три вида <<внутреннего>> преобразования координат.
\subsection*{Преобразование эллипсоидальных координат в геоцентрические}
Входной информацией для преобразования координат являются не только эллипсоидальные координаты \textit{B,~L,~H}, но и параметры эллипсоида, на котором основана система координат: большая полуось эллипсоида \textit{a} и его полярное сжатие \textit{$\alpha$}.
\par
Формулы преобразования эллипсоидальных координат в геоцентрические таковы:
\begin{equation}
\begin{cases}
X = (N + H)cosB cosL
\\
Y = (N+H)cosB sinL
\\
Z = (N(1-e^2)+H)sinB
\end{cases}
\end{equation}

\end{document}