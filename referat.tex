\documentclass[12pt]{book}
\usepackage{ucs}
\usepackage[utf8]{inputenc}
\usepackage[russian]{babel}
\title{Введение}
\author{Алексей Борисов}
\begin{document}
\chapter*{Введение}
\par
Координаты и разности координат, вычисленные по результатам спутниковых измерений, относятся к декартовым геоцентрическим координатам ПЗ-90 (при использовании ГЛОНАСС) или WGS-84 (при использовании GPS). В связи с этим возникает задача пересчёта координат из декартовой системы координат к конкретным системам координат, используемым в инженерно-геодезических работах.
\par
Проблема перехода из одной системы координат в другую (или преобразования координат) рассматривалась в математике неоднократно. В настоящее время насчитывается более десятка различных методов преобразования координат, и все анализировать не имеет смысла, так как каждый разрабатывался под конкретную практическую задачу, как правило, далёкую от проблем геодезии. Для того, чтобы обосновать наиболее целесообразный метод преобразования координат, необходимо на основе анализа свойств существующих систем координат, их особенностей и точностных характеристик геодезических сетей сформулировать основные требования к методу преобразования координат, при этом учесть уникальные точностные возможности результатов спутниковых измерений.
\par
Основные отличия систем координат, применяемых в инженерной геодезии, от декартовой системы координат таковы:
\begin{enumerate}
\item В геодезии, как правило, не используется единая трёхмерная система координат, которую применили в спутниковых навигационных системах. Для определения
\item И так далее.
\end{enumerate}

\end{document}
