\documentclass[12pt]{report}
\usepackage{ucs}
\usepackage[utf8]{inputenc}
\usepackage[russian]{babel}
\usepackage{graphicx}
\usepackage{amsmath}
\title{Введение}
\author{Алексей Борисов}
\begin{document}
\chapter*{2. Преобразование координат}
В настоящей главе будут рассмотрены два вида преобразования координат: <<внутреннее>> преобразование из одной системы в другую  (например, из эллипсоидальной в геоцентрическую) и <<внешнее>> (например, преобразование геоцентрических координат на эллипсоиде WGS-84 в геоцентрические координаты на эллипсоиде Красовского в системе СК-42).
\section*{<<Внутреннее>> преобразование координат}
Рассмотрим три вида <<внутреннего>> преобразования координат.
\subsection*{Преобразование эллипсоидальных координат в геоцентрические}
Входной информацией для преобразования координат являются не только эллипсоидальные координаты \textit{B,~L,~H}, но и параметры эллипсоида, на котором основана система координат: большая полуось эллипсоида \textit{a} и его полярное сжатие \textit{$\alpha$}.
\par
Формулы преобразования эллипсоидальных координат в геоцентрические таковы:
\begin{eqnarray}
X = (N + H)cosB cosL\\
Y = (N+H)cosB sinL\\
Z = (N(1-e^2)+H)sinB,
\end{eqnarray}
где \textit{B, L, H} --- геодезические широта, долгота и высота пункта, \textit{N} --- радиус кривизны первого вертикала, \textit{a} --- большая полуось эллипсоида, \textit{e} --- первый эксцентриситет. Значения  \textit{N} и  \textit{e} вычисляются по следующим формулам:
\begin{eqnarray}
N = \frac {a}{\sqrt {1 - e^2 sin^2 B} }
\\
e^2 = 2\alpha - \alpha^2
\end{eqnarray}
\subsection*{Преобразование геоцентрических координат в эллипсоидальные}
Геоцентрические координаты \textit{X, Y, Z} преобразуются в эллипсоидальные (криволинейные) \textit{B, L, H} по следующим формулам:
\begin{eqnarray}
tg L = \frac{Y}{X},
\\
tgB = \frac{Z}{R} \cdot \frac{r^3 + be'^2  Z^2}{r^3 - be(1-e^2)R^2}
\\
H = R cosB + Z sinB - a\sqrt{1 - e^2 sin^2 B},
\end{eqnarray}
где
\begin{eqnarray}
R = \sqrt{X^2 + Y^2}
\\
r = \sqrt{Z^2 +(X^2 + Y^2)(1 - e^2)}
\end{eqnarray}
\subsection*{Преобразование эллипсоидальных координат в координаты в проекции Гаусса}
Есть несколько формул преобразования эллипсоидальных координат в координаты в проекции Гаусса. Вот один из них.
\begin{eqnarray}
x = X + a_2 l^2 + a_4 l^4 + a_6 l^6 + a_8 l^8, y = b_1 l + b_3 l^3 + b_5 l^5 + b_7 l^7,
\end{eqnarray}
где
\begin{eqnarray*}
a_2 = \frac{1}{2} N sinB cosB
\\
a_4 = \frac{1}{24} N sinB cos^3 B(5 - tg^2 B + 9 \eta^2 + 4 \eta^4)
\\
a_6 = \frac{1}{720} N sinB cos^5 B(61 - 58 tg^2 B + tg^4 B + 270 \eta^2 - 330 \eta^2 tg^2 B)
\\
a_8 = \frac{1}{40320} N sinB cos^7 B(1385 - 3111 tg^2 B + 543 tg^4 B - tg^6 B)
\\
b_1 = N cos B
\\
b_3 = \frac {1}{6} N cos^3 B(-tg^2B + \eta^2)
\\
b_5 = \frac {1}{120}N cos^5 B(5 - 18 tg^2 B + tg^4 B - 14 \eta^2 - 58 \eta^2 tg^2 B)
\\
b_7 = \frac{1}{5040} N cos^7 B (61 - 479 tg^2 B + 179 tg^4 B - tg^6 B)
\\
n = e' cos B
\\
X = a(1 - e^2)(A_X \frac {B''}{\rho} - \frac{B_X}{2} sin 2B + \frac{C_X}{4}sin^4 B)
\\
A_X = 1 + \frac{3}{4}e^2 + \frac{45}{64}e^4
\\
B_X = \frac{3}{4} e^2 + \frac{15}{16}e^4
\\
C_X = \frac {15}{64}e^4
\end{eqnarray*}
\section*{<<Внешнее>> преобразование координат}
Для проведения <<внешнего>> преобразования координат нужно знать семь параметров преобразования:
\begin{itemize}
\item Масштабный коэффициент \textit{m}, м,
\item Три смещения начала координат\textit{$\Delta$ X, $\Delta$ Y и $\Delta$ Z}, м,
\item Три параметра поворота осей координат друг относительно друга \textit{$\omega_x$, $\omega_y$ и $\omega_z$}, рад.
\end{itemize}
Данные параметры актуальны для преобразования как геоцентрических, так и эллипсоидальных координат. Параметры преобразования между некоторыми системами координат (в частности, WGS-84 в ПЗ-90 и ПЗ-90 в СК-42) можно найти в ГОСТ Р 51794---2008 <<Методы преобразований координат определяемых точек>>.
\subsection*{Преобразование геоцентрических координат}
Преобразование геоцентрических координат осуществляется по семипараметрической формуле Гельмерта:
\begin{eqnarray}
\begin{pmatrix}X_2\\Y_2\\Z_2\end{pmatrix}=
\begin{pmatrix}\Delta X_{1-2}\\\Delta Y_{1-2}\\ \Delta Z_{1-2}\end{pmatrix}+
(1 + m) \cdot
\begin{pmatrix}
1 & \omega_Z & - \omega_Y \\
- \omega_Z & 1 & \omega_X \\
\omega_Y & - \omega_X & 1
\end{pmatrix}
\cdot
\begin{pmatrix}
X_1
\\
Y_1
\\
Z_1
\end{pmatrix}
\end{eqnarray}
\subsection*{Преобразование криволинейных координат}
Для осуществления преобразования криволинейных координат требуются те же семь параметров преобразования, что и для преобразования геоцентрических координат. Формулы преобразования таковы:
\begin{equation}
\begin{cases}
B_2 = B_1 + \Delta B,\\
L_2 = L_1 + \Delta L,\\
H_2 = H_1 + \Delta H,
\end{cases}
\end{equation}
где
\begin{eqnarray*}
\Delta B = \frac{\rho}{M + H} \cdot
 ( \frac {N} {a} e^2 sin B cos B \Delta a + (\frac{N^2}{a^2} + 1)N sin B cos B \frac {\Delta e^2}{2} - (\Delta X cos L + \\ + \Delta Y sin L) sin B + \Delta Z cos B ) - \omega_x sin L(1 + e^2 cos2B) + \\ +  \omega_y cosL (1 + e^2 cos 2B) - \rho m e^2 sin B cos B,
\\
\\
\Delta L = \frac {\rho}{(N + H)cos B}(-\Delta X sin L + \Delta Y cos L) + tg B(1 - e^2)(\omega_x cos L + \omega_y sin L) - \omega_z,
\\
\\
\Delta H = - \frac{a}{N} \Delta a + N sin^2 B \cdot \frac{\Delta e^2}{2} + (\Delta X cos L + \Delta Y sin L)cos B +\\+\Delta Z sin B - N e^2 sin B cos B (\frac{\omega_x}{\rho}sin L - \frac{\omega_y}{\rho} cos L)+(\frac{a^2}{N} + H)m,
\\
\\
a = \frac{a_1 + a_2}{2}, \  e^2 = \frac {e^2_1 + e^2_2}{2}, \  \Delta a = a_2 - a_1, \ \Delta e^2 = e^2_2 - e^2_1.
\end{eqnarray*}
\subsection*{Преобразование координат в проекции Гаусса}
Преобразование координат в проекции Гаусса выполняется по следующему алгоритму:
\begin{enumerate}
\item На входе получаются геоцентрические координаты  \textit{X, Y, Z}.
\item  Геоцентрические координаты преобразуются в криволинейные по формулам (6) -- (10).
\item  Криволинейные координаты преобразуются в координаты в проекции Гаусса по формулам (11).
\item  Определяются разности вычисленных координат между пунктами с номером \textit{i} и опорным пунктом:
\begin{eqnarray}
\Delta x_i = x_i - x_0,  \Delta y_i = y_i - y_0
\end{eqnarray}
\item Выбираются два желательно наиболее удалённых друг от друга пункта и вычисляются
\end{enumerate}
\end{document}